\documentclass[../main.tex]{subfiles}
\begin{document} 
Lung cancer is the leading of cause and responsible for 19\% of cancer-related deaths worldwide - it is the most commonly diagnosed cancer, with 2.5 million new cases reported in 2022 \cite{hendriksNonsmallcellLungCancer2024}.  Despite combination therapies (two or more anti-cancer drugs applied as part of structured treatment) providing superior clinical outcomes \cite{duarteEvaluationSynergismDrug2022} to single drug treatments, many possible combination therapies remain underexplored due to labour-intense and expensive experimental setups. 

Such drug synergy screens are relatively simple to carry out in 2D cell cultures; however these are not representative of more 3D cell culture models and could partially explain high failure rates of clinical trials \cite{folkessonHighthroughputScreeningReveals2020} \cite{barrera-rodriguezMultidrugResistanceCharacterization2015}. More advanced three-dimensional cell culture models can better replicate key features of real tumours, such as cell-to-cell adhesion or blood flow, which play key roles in treatment efficacy \cite{guntiOrganoidSpheroidTumor2021}. However, these approaches are state-of-the-art and hard to scale to truly high-throughput testing.

This project aims to address these challenges by combining recent advances in microfluidic technologies with tumour spheroid models to enable faster and more structured screening of combination therapies. By improving the efficiency with which multiple drugs can be tested together, this approach could help identify more effective treatment strategies and support future advances in personalised cancer therapy. Lastly, a tailored mathematical modelling approach approach is proposed to further increase the platforms screening efficiency;, thus introducing a feasible way to screen combinatorial treatment of 3 or more drugs and applications to precision medicine.
\end{document}