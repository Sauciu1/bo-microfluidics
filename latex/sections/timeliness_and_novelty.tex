\documentclass[../main.tex]{subfiles}
\begin{document} 
Combinatorial treatment strategies have been repeatedly shown to outperform the application of single drug for cancer treatment and hold great potential for treatment of NSCLC \cite{kaurPromisingCombinatorialTherapeutic2024}, \cite{desaiImmunotherapybasedCombinationsMetastatic2023}. The structured application of drugs allows to steer the tumour evolution thus promoting targeted vulnerabilities in the tumour cell population non affected by a prior drug in the series. This approach provides a solution to developed drug-resistance and high toxicity of combined-drug strategies. However, despite improved treatment outcomes, and rapidly mounting number of newly discovered drugs, development of new multi-phase campaigns remains sporadic and non-systematic. 
Include:


Recent explosion in machine learning and high throughput screening has significantly accelerated cancer drug discovery rate. In-silico tools, such as docking simulations, structure-based supervised leaning, or protein folding, allow to identify salient targets for further wet-lab testing at a comparatively low cost. This computational screening is then often combined with automated screening of drug efficacy using various in vitro models (e.g. organoids or tumour spheroids on Microfluidic chips.
\end{document}