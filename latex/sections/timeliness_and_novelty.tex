\documentclass[../main.tex]{subfiles}
\begin{document} 
Approximately 2.5 million new lung cancer cases are diagnosed annually, 85\% of which are non-small-cell lung cancer (NSCLC) \cite{brayGlobalCancerStatistics2024}. Notably, 25\% of NSCLC cases occur in lifetime non-smokers, and lung cancer in never-smokers is estimated to be the fifth leading cause of cancer-related death worldwide \cite{hendriksNonsmallcellLungCancer2024}. These figures are expected to rise further, with total annual cancer diagnoses projected to increase from 20 million to 35 million by 2050 \cite{hendriksNonsmallcellLungCancer2024}.

Combinatorial treatment strategies have been repeatedly shown to outperform the application of single-agent treatment and hold great potential for treatment of NSCLC \cite{kaurPromisingCombinatorialTherapeutic2024} \cite{desaiImmunotherapybasedCombinationsMetastatic2023a}. Combining drugs with different modes of action allows to target the heterogenous tumour cell population if NSCLC \cite{hendriksNonsmallcellLungCancer2024} \cite{kaurPromisingCombinatorialTherapeutic2024}. The approach provides a solution This approach can provide a solution to evolving tumour resistance and reduce toxicity by avoiding high single drug doses \cite{kaurPromisingCombinatorialTherapeutic2024}. However, despite improved treatment outcomes, and rapidly mounting number of newly discovered drugs, development of new multi-phase campaigns remains sporadic and non-systematic \cite{tanSystematicReviewCombinations2021}. Moreover, the explosion of machine learning tools for drug discovery, such as docking-simulations, structured-based virtual screening, or protein folding is bound to increase the relative fraction of unexplored combination therapies.

Unlike existing spheroid screening platforms, this work integrates 3-inlet microfluidic concentration gradient generators directly with physiologically perfused tumour spheroid arrays, enabling dense, equipment-efficient mapping of multi-drug dose-response landscapes within a single chip. Moreover, a structured-batch Bayesian optimisation algorithm is explicitly matched to the 3-inlet CGG, addressing physical batching constraints ignored by conventional BO. The co-design should enable efficient exploration of high-dimensional drug spaces that are otherwise infeasible with grid-based or unconstrained optimisation approaches.
\end{document}