\documentclass[../main.tex]{subfiles}
\begin{document} 
Combinatorial cancer therapies offer a substantially larger therapeutic design space than single-drug treatments, but systematic exploration of this space is limited by experimental throughput and cost. In particular, screening drug combinations involving three or more agents rapidly becomes infeasible using conventional experimental designs due to the curse of dimensionality. Consequently, there is a need for integrated experimental and computational platforms that can efficiently explore high-dimensional drug-dose response landscapes. 

To address this challenge, we propose combining the unique features of two high-throughput screening platforms - mass-generation of tumour spheroid arrays under close-to-physiological flow conditions \cite{princeMicrofluidicArraysBreast2022} and generation of two dimensional concentration gradients \cite{shenConstructionMultipleConcentration2023}. CGGs allow generating discrete concentration gradients within a single chip, thus reducing equipment and experimental overhead \cite{wuReviewDevelopmentApplication2024}. Tumour spheroid arrays recapitulate key aspects of the tumour microenvironment (TME), including cell–cell interactions, ECM matrix, and flow shear conditions, while enabling high-content imaging of large numbers of uniform spheroids \cite{guntiOrganoidSpheroidTumor2021} \cite{leeRecentAdvancesApplications2024}.


Possible drug-dose response combinations of 4 or more drugs provide a screening space too large for conventional techniques (usually grid search). Consequently, a  modifications of Bayesian Optimisation (BO), a black box parameter exploration, is proposed. The lab-in-a-loop workflow maximise sample efficiency by rapidly identifying high-synergy regions that are not self-evident from single-drug dose response trials or biochemical pathways.

Furthermore, the active learning model paired with a standardised platform, would enable transfer learning. Knowledge acquired from generic cell-line screens can serve as informative priors, enabling rapid and adaptive re-evaluation of the most salient drug combinations using patient-derived cells under constraints of limited material and time.
\end{document}