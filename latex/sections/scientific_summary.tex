\documentclass[../main.tex]{subfiles}
\begin{document} 
Combination therapies for cancer, rely on evolutionary guiding tumour cell populations to maximise the combined effect of chemotherapy drugs applied in series. Such approaches have been repeatedly shown to provide better patient outcomes via (increased efficacy and decreasing off-site toxicity) and are defined by higher than additive therapeutic index that would be expected from combining the effect of each individual drug. Despite their benefit, sequential combination therapies are underexplored compared to concurrent combination therapies (drugs applied simultaneously).

At its heart, the research proposal focusses on combining 2 rapidly advancing technologies: high-throughput tumour spheroids and efficient parameter exploration models into a lab-in-a-loop platform capable of exploiting the unrealised gains within the research gap and providing new avenues for personalized medicine.

Tumour spheroids are just a step bellow in-vivo representativeness of animal models, bellow organoid chips, but are relatively inexpensive to produce en masse. They have been successfully used for screening of combinatorial drugs and can be rapidly cultured, thus allowing quick iteration cycles.  

Modifications of Bayesian Optimisation (BO), a black box parameter exploration, would allow to maximise sample efficiency, thus decreasing experimentation cost and making such exploration feasible. Sequential screening (compared to simultaneous screening), provides a much larger parameter space to explore, where emerging drug dose delivery sequence,  duration, and timings invoke the curse of dimensionality and thus make conventional screening approaches unfeasible. More importantly, if combined with standardized platform, BO allows for integration of prior information, thus massively accelerating testing.

Lastly, due to standardised properties of the system and BO suitability to transfer learning, the platform could be easily adapted for precision medicine. The explored sequential therapy space provides a priori, which would allow to rapidly and adaptively reevaluate the most salient therapy combinations with patient cells.
\end{document}